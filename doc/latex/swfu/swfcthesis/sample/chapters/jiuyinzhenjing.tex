\chapter{九陰真經}
《九陰真經》\footnote{参见维基百科 -
  \href{http://zh.wikipedia.org/wiki/\%E4\%B9\%9D\%E9\%99\%B0\%E7\%9C\%9F\%E7\%B6\%93}{九陰真經}}是
金庸小说中虛構的武學巨著,乃金庸武俠小說系列中最負盛名也最具傳奇性的武學。與《九陽真經》齊名
\cite{jyinzj}。 

\section{傳奇}

\subsection{成書經過}

\subsubsection{初版}
在《射鵰英雄傳》第一版中,《九陰真經》和《九阳真经》原是相傳是達摩祖師所寫下,話說達摩東來,與中土武士較技,雙方互有勝負;面壁九年後參透了武學精奧,寫成二書。
\subsubsection{二版、新修版}
宋徽宗於政和年間,下旨命令一個聰明的刻書人——黃裳刻寫一本道家書籍《萬壽道藏》,共分五千四百八十一卷。由於是為皇帝刻書,黄裳怕出錯殺頭,他花了好幾年小心校對,不知不覺便精通了道學,還領悟了當中的武功。經過慢慢的修練,成了武功高手。

不久,宋徽宗下旨要黄裳派兵去剿滅魔教——明教。不料,當中高手如雲,黄裳遂親自向明教高手挑戰,一口氣殺了
幾個法王使者。哪知道他所殺的人中,有幾個是武林中名門大派的弟子,於是被各大派尋仇,黄裳寡不敵眾,受傷
後逃亡,躲在不毛之地。仇家隨後將他家裡的父母妻兒殺了個乾乾淨淨。為了避免再受追殺,記下了的敵人招式,
苦思破解方法。當他想通了,已經過了四十多年,仇家全都死了。他便覺得自己時日無多,便把畢生心血,寫成上
下两卷的《九陰真經》。

\subsection{重現華山}
天下五絕,即黃藥師、歐陽鋒、段智興、洪七公和王重陽,同意誰勝出誰獨得《九陰真經》。經過七日七夜的大戰,由王重陽奪得。在他臨死前,把《九陰真經》交給周伯通。
\subsection{刻於古墓}
王重陽得知林朝英創出剋制全真教武功的《玉女心經》,不禁佩服起來。後來王重陽奪得《九陰真經》,於是把其中部份武学刻於古墓中,來剋制《玉女心經》。後為楊過、小龍女師徒習得。
\subsection{被人偷竊}
後來,周伯通往桃花島見黃藥師,警告他有人會偷《九陰真經》。黃藥師的妻子馮蘅有機會細閱《九陰真經》的下半部,憑著過目不忘的能力已經把全部內容記下來,再自行書寫出一本交予黃藥師。

不久,黃藥師的徒弟梅超風和陳玄風背叛了黃藥師,偷走了該本《九陰真經》,離開桃花島。黃藥師一怒之下,把所有弟子打跛。身懷六甲的馮蘅為了黃藥師,再度默寫《九陰真經》下卷,但是因為離上次背經已有一段時日,經文已多半忘記了。又因為她努力回想經文導致身心俱疲,生下黃蓉不久就去世了。

另外,陳梅二人離開桃花島後,由於陳梅二人不懂玄門道學,因此陳玄風憑著自己的臆測解經,再轉授梅超風,在
曲解經義的情形下,兩人以「五指插入人的頭蓋骨」或「服用砒霜之毒」等方式,練成了陰毒的「九陰白骨爪」和
「摧心掌」,成了江湖上惡名昭彰的「黑風雙煞」。兩人橫練功夫甚高,不怕擊打,只對利器稍有忌憚。但各自有
個脆弱無比的「罩門」,一碰即死。陳玄風的罩門在肚臍,梅超風的則在舌下。蒙古荒山夜戰中,陳玄風雖以九陰
白骨爪和摧心掌殺了江南七怪的張阿生,卻也被當時年僅六歲的郭靖用匕首刺中罩門而死。梅超風則輾轉到了金國
王府,收楊康為徒,並將九陰白骨爪傳給了楊康。

\subsection{落入郭靖手中}
後周伯通為了作弄郭靖,將九陰真經教給了他,卻不告訴他學的是九陰真經,間接造成黃藥師對郭靖的誤會。
\subsection{藏在倚天劍}
後來郭靖与黃蓉知道襄陽終不可守,將楊過的玄鐵重劍熔了,再加以西方精金,鑄成了一柄屠龍刀,一柄倚天劍,並把《武穆遺書》藏在屠龍刀中,把《九陰真經》與九陰速成之法、《降龍十八掌掌法精要》藏在倚天劍中。 倚天劍被交與女兒郭襄手中,而郭襄則為娥眉派的創始人,後傳承侄滅絕師太,最後落與周芷若手上,屠龍刀下落不明。 多年後於元順帝年間,峨嵋派掌門周芷若放逐汝陽王之女趙敏後,將倚天劍及屠龍刀互砍,取得兵書和武功秘笈而學到九陰白骨爪。其九陰真經內力在身受玄冥神掌之傷後,被張無忌以九陽真經內力醫治時化去大部份。 在新修版中改為兩塊玄鐵鐵片,一為桃花島所在地,一為桃花島地圖,《九陰真經》、《降龍十八掌掌法精義》和《武穆遺書》放在桃花島中心,九陰真經改為完整版。因為金庸感到把書本放到刀劍的夾層中並不合理,鑄造刀劍時會燒去,因此修改為鐵片。
\section{經文選輯}

\begin{itemize}
\item 「天之道,損有餘而補不足,是故虛勝實,不足勝有餘。」
\item 「人徒知枯坐息思為進德之功,殊不知上達之士,圓通定慧,體用雙修,即靜而動,雖攖而寧。」
\end{itemize}

\section{記載神功}

\subsection{上卷(內功)}
\begin{description}
\item[易筋鍛骨篇] 练成后功力等方面均进展迅速。内容提到:「人徒知枯坐息思为进德之功,殊不知上达之士,圆通定慧,体用双修,即动而静,虽撄而宁。」不但有打坐修炼的静功,也有由外而内的动功。
\item[療傷篇] 療傷篇系為療傷之用,亦能用以增加功力。由於能練九陰真經者已有一定修為,故療傷對於一般的外傷亦不多提,主要是談及內傷治方面。
\item[點穴篇] 此篇隻述及點穴方面的要旨,未見有詳細的招式。
\item[解穴秘訣] 为自通穴道之法,可在被人点中穴道或闭塞时,即可用此法自行打通。(《神雕俠侶》第十九回《重陽遺刻》)
\item[移魂大法] 为摄心术的一种,实质有如现代的催眠,能用来对付武功高强,但心志不坚的对手。(君山大會中黃蓉以移魂大法克制彭长老的慑心术)
\item[蛇行狸翻] 即便在地上翻滚,也是灵动异常。(《射鵰英雄傳》第六十二回《阴错阳差》中,周伯通以“蛇行狸翻”避過黃藥師的追擊。)
\item[閉氣秘訣] 运用此法即可长时间不呼吸。(《神雕俠侶》第十九回《重陽遺刻》)
\item[飛絮勁] 《射鵰英雄傳》第三十八回《錦囊密令》中,郭靖以“飞絮劲”化解歐陽鋒一招。
\item[總綱] 此篇為九陰真經的總綱及上卷的最後一章。
以梵文譯音寫成,初版作者為達摩,自無問題,二版卻成為黃裳為免九陰真經落入歹人之手而加防備的一種手段。九陰真經總綱精奧無比,能將修真之士所遇的幻象之類,轉為神通。
新版九陰真經總綱更糾正了道家武學偏重陰柔的流弊,實現了陰陽互濟、剛柔並重的武學最高境界。  
\end{description}
\subsection{下卷(武功)}
\begin{description}
\item[摧心掌] 中此掌者,外在并无任何伤痕,但内裡的五脏六腑已然碎裂。
\item[白蟒鞭法] 使用極長的白蟒鞭,如靈蛇出洞,伸縮自如,靈動之極。
\item[大伏魔拳] 稳实刚猛之气的掌法,招数神妙无方,拳力笼罩之下威不可挡。 (《神鵰俠侶》中,周伯通以大伏魔拳與楊過的「黯然銷魂掌」對拆)
\item[收筋縮骨法] 君山大會中郭靖以此“收筋缩骨法”脫縛。(《射雕英雄傳》第二七回\cite{shediao})
\item[摧堅神爪] 五指发劲,无坚不破,摧敌首脑,如穿腐土,出爪时爪心有强大的吸力可隔空取物或吸取他人功力,爪指有强大的透劲可隔空伤人,一收一放,一开一合,合乎武学大道之理。
\item[九阴白骨爪] 受此功夫死亡者头顶五个指洞,是极为阴险的工夫,但其实为一正而不邪的功夫,九阴白骨爪
  在《九阴真经》中原是“摧堅神爪”,「铜尸」陈玄风、「铁尸」梅超风学不到《九阴真经》上半部中养气归元、
  修习内功的心法,但凭已意,胡乱揣摸,不知“摧敌首脑”是「攻敌要害」之意,以为是以五指去插入敌人头盖,
  又以为练功时必须如此,硬是把上乘武功练到了邪路上。和峨眉派掌门周芷若都为求速成,亦练得此功,夺得武
  功天下第一的名头。

  註:二版中,九陰白骨爪為黑風雙煞錯練九陰神爪而成,真經中本無白骨爪;新三版增寫九陰白骨爪等功為黃裳
  之敵的武功,黃裳弟妹便死於此功之手,黃裳便創摧堅神爪以破白骨爪。另二版中本無白蟒鞭法,只有梅超風的
  兵器毒龍鞭及真經中一套無名鞭法;新三版中毒龍鞭易名白蟒鞭,並加上白蟒鞭法。 
\end{description}

%%% Local Variables: 
%%% mode: latex
%%% TeX-master: "../sample"
%%% End: 
