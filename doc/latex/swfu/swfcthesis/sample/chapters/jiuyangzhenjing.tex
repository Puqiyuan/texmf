\chapter{九陽真經}
《九陽真經》\footnote{参见维基百科 - \href{http://zh.wikipedia.org/wiki/\%E4\%B9\%9D\%E9\%99\%BD\%E7\%9C\%9F\%E7\%B6\%93}{九陽真經}}是金庸小說中虛構的武學巨著\cite{jyangzj}。

《九陽真經》是這本內功秘集的名稱。內功練成,便名《九陽神功》,乃是金庸武俠小說系列中極強、甚至最強的內功,被喻為非任何內功所能比。與《九陰真經》齊名。

金庸武俠小說系列中練成全套《九陽神功》的人物,除了創者,就只有《神鵰俠侶》、《倚天屠龍記》中的僧人覺
遠,以及《倚天屠龍記》的主角張無忌練成。

\section{成書經過}
在《倚天屠龍記》第一版中,《九陽真經》與《九陰真經》相輔相成,同是達摩所寫下。《九陰真經》有無數神妙武功,《九陽真經》雖然只有內功,但神功大成後,卻非世上的任何武功所能傷害。

二版中,《九陰真經》變成黃裳所創,《九陽真經》則是相傳是達摩祖師所寫下的,但後來張君寶悟到達摩祖師是天竺人,就算會寫中華文字,也必文理粗疏,九陽真經文字佳妙,外國人決計寫不出,定是後世中土人士所作。多半便是少林寺中的僧侶,假托達摩祖師之名,寫在天竺文字的《楞伽經》夾縫之中。

在最新修改版本中,作者寫成在嵩山中一位奇士鬥酒勝了王重陽,得以借觀《九陰真經》,此人觀看後覺得《九陰真經》陰氣太重,一味崇揚老子之學,只重以柔克剛,以陰勝陽,未及陰陽互濟之妙,於是在四卷梵文《楞伽經》的行縫之中,寫下自創的《九陽真經》。

然而,新修版《射鵰英雄傳》的《九陰真經》總綱卻明言「九陰極盛」乃是災害,總綱的要旨亦是要糾正這一毛病,故此奇士所見,應不包括《九陰真經》的梵文總綱。《九陰真經》的梵文總綱與九陽真經相同,皆為「陰陽互濟」之道。若修習成功,便與九陽真經達到相同的「武學最高境界」。
\section{重見天日}
於《神鵰俠侶》中,楊過打敗金輪法王(新修版改稱「金輪國師」)後,蒙古武士尹克西和瀟湘子逃到嵩山,後發
現張君寶(後來的張三丰)在少林寺廊下讀《楞伽經》,尹克西悄悄走到他身後,伸手點了他的穴道,把那四卷
《楞伽經》取去。


\section{經在猴中}
後來,張君寶的師父覺遠追尹克西和瀟湘子到華山,巧遇楊過、小龍女和郭襄。尹克西和瀟湘子眼看無法脫身,剛
好身邊有只蒼猿,兩人心生一計,便割開蒼猿肚腹,將經書藏在其中。後來覺遠、張君寶、楊過等搜索瀟湘子、尹
克西二人身畔,不見經書,便放他們帶同蒼猿下山。後來瀟湘子和尹克西帶同蒼猿,遠赴西域,兩人心中各有所忌,
生怕對方先習成經中武功,害死自己,互相牽制,遲遲不敢取出猿腹中的經書,最後來到崑崙山的驚神峰上,尹湘
兩人互施暗算,鬥了個兩敗俱傷而死。這部修習內功的無上心法,從此留在蒼猿腹中。

\section{覺遠傳經}

尹克西臨死時遇見「崑崙三聖」何足道,良心不安,請他赴少林寺告知覺遠大師,那部經書是在這頭蒼猿的腹中。但他說話之時神智迷糊,口齒不清,他說「經在猿中」,何足道卻聽做「金在油中」。何足道信守然諾,果然遠赴中原,將這句「金在油中」的話跟覺遠說了。覺遠無法領會其中之意,固不待言,反而惹起一場絕大的風波,令覺遠、張君寶被少林寺所逐。覺遠圓寂前,念起《九陽真經》,令郭襄、張君寶和少林寺羅漢堂首座無色禪師各有所悟,無色得其高(因三人中武功最高並與其本身武功印證),郭襄得其博(因家學淵博,所學甚廣),張君寶得傳承最多,後來郭襄、張君寶出家開創峨嵋、武當,各以此為基礎,創出了少林九陽功、峨嵋九陽功、武當九陽功,共三脈《九陽功》。

後來《倚天屠龍記》中張無忌因朱子柳的後代朱長齡的掀連意外地進入崑崙山一山谷,巧遇當年的蒼猿,因好心治
療的其肚腹上的傷痕,取出內藏之經書,得以習得完整的九陽神功,不但纏綿體內的寒毒盡皆驅除,內力更提升至
絕高的境界。待得到乾坤一氣袋的奇遇後,張無忌的九陽神功方克大成。

\section{永埋土中}
在張無忌走進崑崙山山谷後五年,他學全了九陽真經,於是把四卷載有《九陽真經》的《楞伽經》、以及胡青牛的
《醫經》、王難姑的《毒經》埋在地洞裡。

\section{經文選輯}
\begin{itemize}
\item 「他強任他強,清風拂山崗;他橫任他橫,明月照大江。」
\item 「他自狠來他自惡,我自一口真氣足。」
\end{itemize}

\section{龍虎門的九陽神功}

在龍虎門,主角王小虎依此絕學而踏入頂級高手的領域,而白蓮教教主東方無敵的九陽神功乃是練的最出神入化的
一個。自古以來,江湖傳言「九陽神功驚俗世」,「君臨天下易筋經」,這兩套絕學號稱頂級絕學。然而,東方無
敵武學智慧卻突破原有九陽神功的範疇,成就九陽五絕合一。武學領域已達至前無古人的十陽聖火,媲美黑級高階
功力甚至超越黑級高階。白蓮教是東方家族企業,由於無懼未指派繼承人就駕崩,故無忌及無敵以武論尊,最終無
敵擊敗無忌就任白蓮教主,所以東方無敵與火雲邪神及文珠天尊不同的地方就在於他必須靠自己的努力,沒有其他
人可依靠,羅剎教的第二號人物是老邪神,通天教的第二號人物則是文珠天尊自己,因為第一號人物是老天尊,白
蓮教就缺乏與他們並駕齊驅的強者。雄才大略的東方無敵向羅剎教挖角金羅漢成為日聖使,同時也向龍虎門挖角王
風雷擔任月聖使,這新的鐵三角,遠比之前日月聖使還強,甚至在《王風雷傳》中號稱「無敵組合」,而《新著龍
虎門》中,白蓮教也相當聰明,不跟強橫的龍虎門為敵。

龍虎門的九陽神功分內功和外功\footnote{(按:此處非金庸小說,乃他書中雷同之武學名稱。)}。

\subsection{內功:九陽真經}

\begin{description}
\item[九陽神功之十陽聖火:]九陽神功第十陽突破原有九陽神功的範疇,媲美黑級高階功力甚至超越黑級高階。
\item[九陽神功之九陽合一:]東方無敵在黑龍就任羅剎教主的登位大典上使用的「九陽合一」,所發出的「九陽
  大霹靂」威力強橫,三皇全力發出「九陰大霹靂」也只能拼成平手,而這個戰績已經媲美神山之役時火雲邪神的
  易筋經黑級高階功力。這些是不是就說明「九陽合一」被設定為媲美黑級高階功力呢? 
\item[九陽神功之九陽歸一:]發揮九陽的顛峰境界,可令神功進入另一層次。
\item[九陽神功之九陽啟泰:]號稱九陽神功頂級功力。
  \begin{multicols}{2}
    \begin{enumerate}
    \item 第九陽百會穴
    \item 第八陽至陽穴
    \item 第七陽脊樑穴
    \item 第六陽手少陽三焦經穴
    \item 第五陽手太陽小腸經穴
    \item 第四陽足太陽膀胱經穴
    \item 第三陽足陽明胃經穴
    \item 第二陽丹田穴
    \item 第一陽心坎穴
    \end{enumerate}
  \end{multicols}
\end{description}

\subsection{外功:九陽五絕}

\begin{description}
\item[第一絕:]霹靂神掌[九陽霹靂]:九陽五絕中威力最強最驚世駭俗的掌法。只有降龍掌可與之一拚。 
  \begin{enumerate}
  \item 九陽小霹靂:練到九陽是可以控制方向的,練到八陽就只能當普通的直線型氣功炮。
  \item 九陽大霹靂:彷彿超巨能靈彈,威力無堅不摧,「火雲邪神」曾敗於此招下。
  \item 雙霹靂合璧:風雷傳裡凶星助無敵升級,無敵打出的霹靂威力可能接近雙陽霹靂合璧的威力,這招號稱天
    上無人能擋。
  \end{enumerate}
\item[第二絕:]九陽神劍:用手指發出劍氣傷敵;最強招「雙陽劍合璧」,需第八陽以上功力方可使用。
  \begin{multicols}{3}
    \begin{enumerate}
    \item 大商劍
    \item 少商劍
    \item 大衝劍
    \item 少衝劍
    \item 大澤劍
    \item 少澤劍
    \item 大陽劍
    \item 少陽劍
    \item 雙陽劍
    \end{enumerate}
  \end{multicols}
\item[第三絕:]陰陽大挪移:可卸開敵人招式,有外傷時不可使用,否則血流不止。
\item[第四絕:]火雲掌:掌法絕學。白蓮教死對頭羅剎教教主學得此掌法,故號稱「火雲邪神」以羞辱白蓮教。[來源請求]
  \begin{multicols}{3}
    \begin{enumerate}
    \item 火雲鐵桶
    \item 火蛇吐信
    \item 火龍穿山
    \item 火雲蓋頂
    \item 火海無邊
    \item 地火燎原
    \item 天火焚城
    \end{enumerate}
  \end{multicols}
\item[第五絕:]烈陽刀:以掌為刀,招法剛猛
  \begin{multicols}{2}
    \begin{enumerate}
    \item 第一式:烈陽普照
    \item 第二式:烈陽雙暉
    \item 第三式:烈陽破頂
    \item 第四式:烈陽焦土
    \item 第五式:烈陽焚天
    \end{enumerate}
  \end{multicols}
\item[五絕合一:]東方無忌 (王風雷傳十二期對東方無忌使用的連續技,並在三十九期命名)
\end{description}


%%% Local Variables: 
%%% mode: latex
%%% TeX-master: "../sample"
%%% End: 
