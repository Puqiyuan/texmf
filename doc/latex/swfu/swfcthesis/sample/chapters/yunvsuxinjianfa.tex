\chapter{玉女素心剑法}
玉女素心剑法\footnote{参见维基百科 - \href{http://zh.wikipedia.org/wiki/\%E7\%8E\%89\%E5\%A5\%B3\%E7\%B4\%A0\%E5\%BF\%83\%E5\%89\%91\%E6\%B3\%95}{玉女素心剑法}},古墓派剑法,属于《玉女心经》中最后一章的武功,古墓派创派祖师林朝英所创\cite{ynxj}。

当年林朝英创下“玉女心经”,虽是要克制全真派武功,但对王重阳始终情意不减,写到最后一章时,幻想终有一日能与意中人并肩击敌,因之“玉女心经”最后一的玉女素心剑法是一对情侣同使,一个使“玉女心经”,一个使全真功夫,相互应援,分进合击。二人同使时,招式名称相同,但招式本身却是大异,相互呼应配合,所有破绽全被旁边一人补去。这路剑法每一招中均含着一件韵事,或“抚琴接箫”,或“扫雪烹茶”,或“松下对弈”,或“池边调鹤”,均是男女与共,说不尽的风流旖旎。只是實際上功夫冠絕天下的林朝英與王重陽根本已無必須兩人合作才能擊退的敵手,所以林朝英始終未有機會實際應用之。

剑法的奥妙在于双剑合壁,男女二人如果是情侣才能体会。合使的两人若是朋友,则相互间心灵不能沟通而太过客气;是尊长小辈又不免照拂仰赖;如夫妻同使,其中含情脉脉、亦甜亦苦的心情却不如热恋中的情侣。只有相互眷恋、未结丝萝的情侣,才能领会林朝英这套“玉女素心剑”心息相通之意。

小龙女从老顽童周伯通处以“养蜂术”换得“左右互搏”后以“左右互搏”催动“玉女素心剑”却又是另一番景象。后小龙
女于全真教大战时又悟出以“天罗地网势”施展“左右互搏”版“玉女素心剑”使得全真七子无话可说。

%%% Local Variables: 
%%% mode: latex
%%% TeX-master: "../sample"
%%% End: 
