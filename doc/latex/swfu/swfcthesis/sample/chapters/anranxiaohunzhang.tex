\chapter{黯然銷魂掌}
黯然銷魂掌
\footnote{参见维基百科 - \href{http://zh.wikipedia.org/wiki/\%E9\%BB\%AF\%E7\%84\%B6\%E9\%8A\%B7\%E9\%AD\%82\%E6\%8E\%8C}{
    黯然銷魂掌}}是金庸武俠小說《神雕俠侶》中的武功,為小說男主角「神雕大俠」、「西狂」楊過獨創的武功。

\section{簡述}

黯然銷魂掌集天下五絕及其他武林高手的絕世武學,包括「中神通」王重陽的全真教內功和《九陰真經》、「東邪」黃藥師的彈指神通和玉簫劍法、「西毒」歐陽峰的蛤蟆功和逆轉經脈、「北丐」洪七公的打狗棒法,以及林朝英的《玉女心經》而成。

這掌法取名自江淹《別賦》中一句:「黯然銷魂者,唯別而已矣」。

至於「相思無用,惟別而已,別期若有定,千般煎熬又何如,莫道黯然銷魂,何處柳暗花明。」一句, 則是2006年黃曉明版神雕俠侶之編劇想出來的,這與江淹的別賦無關。

黯然銷魂掌共有十七式:

這一路掌法,楊過主要在與「老頑童」周伯通、「東邪」黃藥師和金輪法王相鬥時使用。單論掌力,當世唯有郭靖的降龍十八掌可以比擬,而黃藥師的彈指神通亦跟這路掌法不分上下。但使用其掌法之人心情必須黯然,若情境不符則無法發揮其威力。

\begin{description}
\item[心驚肉跳] 单臂负后,凝目远眺,脚下虚浮,胸前门户洞开,全身姿式与武学中各项大忌无不吻合,他小腹肌肉颤动,同时胸口向内一吸,倏地弹出,以胸肌伤人。
\item[杞人憂天] 挥平而上,仍是只用左臂,抬头向天,浑若不见,呼的一掌向自己头顶空空拍出,手掌斜下,掌力化成弧形,四散落下。
\item[無中生有] 手臂下垂,绝无半点防御姿式,待得對手拳招攻到近肉寸许,突然间手足齐动,左掌右袖、双足头锤、连得胸背腰腹尽皆有招式发出,无一不足以伤敌,瞬息之间,【十余招数同时攻到】,说来“无中生有”只是一招,中间实蕴十余招变式后着,
\item[拖泥帶水] 右手云袖飘动,宛若流水,左掌却重滞之极,便似带着几千斤泥沙一般,右袖是北方癸水之象,左拳“是中央戊土之象,轻灵沉猛,兼而有之”。
\item[徘徊空谷] 失传。
\item[力不從心] 失传。
\item[行屍走肉] 踢出一脚,这一脚发出时恍恍惚惚,隐隐约约,若有若无。
\item[魂牽夢縈] 失传。
\item[倒行逆施] 突然头下脚上,倒过身子,拍出一掌,三十七般变化,这掌法逆中有正,正反相冲,自相矛盾,不能自圆其说。
\item[廢寢忘食] 失传。
\item[孤形隻影] 失传。
\item[飲恨吞聲] 失传。
\item[六神不安] 失传。
\item[窮途末路] 失传。
\item[面無人色] 虽是一招,其实中间变化多端,脸上喜怒哀乐,怪状百出,敌人一见,登时心神难以自制,我喜敌喜,我忧敌忧,终至听命于我,此乃无声无影的胜敌之法。
\item[想入非非] 失传。 
\item[呆若木雞] 失传。
\end{description}

%%% Local Variables: 
%%% mode: latex
%%% TeX-master: "../sample"
%%% End: 
