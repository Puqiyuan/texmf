\chapter{倚天屠龍功}
倚天屠龍功
\footnote{参见维基百科 - \href{http://zh.wikipedia.org/wiki/\%E5\%80\%9A\%E5\%A4\%A9\%E5\%B1\%A0\%E9\%BE\%8D\%E5\%8A\%9F}{
    倚天屠龍功}}為金庸武俠小說《倚天屠龍記》中武當派的武功,是張三豐心傷三徒俞岱岩被折斷四肢,而在夜
裡揮手演繹「武林至尊、寶刀屠龍。號令天下,莫敢不從。倚天不出,誰與爭鋒」二十四字,無意間創出一套極高
明的武功,但在七名徒兒中僅有熟練書法且正巧看到張三豐練拳的張翠山習得\footnote{出自倚天屠龍記第四章 字作喪亂意彷徨}。

\section{介绍}

雖然倚天屠龍功中有各兩個「不」字,兩個「天」字,但兩字寫來形同而意不同,招式變化迥異,張翠山在習得倚天屠龍功後,也用了大半天思索融會,把這二十四字練到爛熟於心,不但能空手施展,也可用判官筆、銀鉤當兵刃使出。張翠山曾憑這套武功輕取龍門鏢局三個鏢師、少林派的圓音跟圓業等四名師兄弟,並用這幅字逼得謝遜認輸,保住自己跟殷素素的性命\footnote{出自倚天屠龍記第六章 浮槎北溟海茫茫}。

後來張翠山攜妻子殷素素、兒子張無忌從冰火島回到中原時,在往武當山路上力鬥高麗青龍派高手泉建男,即用判官筆展開倚天屠龍功把他擊敗。\footnote{出自倚天屠龍記第九章 七俠聚會樂未央}可惜這路絕學隨張翠山自刎後淹沒,張三豐也沒再傳給其他弟子。

%%% Local Variables: 
%%% mode: latex
%%% TeX-master: "../sample"
%%% End: 
