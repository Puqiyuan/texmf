\chapter{降龍十八掌}
降龍十八掌\footnote{参见维基百科 - \href{http://zh.wikipedia.org/wiki/\%E9\%99\%8D\%E9\%BE\%8D\%E5\%8D\%81\%E5\%85\%AB\%E6\%8E\%8C}{降龍十八掌}}是金庸武俠小說《天龍八部》、《射鵰英雄傳》、《神雕侠侣》\cite{shendiao}和《倚天屠龍記》\cite{yitian}中丐幫二大謢幫神功之一,降龍十八掌分為十八式。

世紀新修版的《天龍八部》中,則改為降龍廿八掌,對應天上二十八星宿,後來才由乔峰及虛竹簡化為十八掌。

\section{簡介}
降龍十八掌講究剛柔並濟,當剛則剛,當柔則柔,轻重刚柔随心所欲,刚劲柔劲混而为一,劲力忽强忽弱,忽吞忽吐,从至刚之中生出至柔,天下阳刚第一,掌法之妙,天下无双,招招须用真力,说是外门武学中的巅峰绝诣,動作雖似簡單無奇,但掌掌現神龍,招招威力無窮,招式简明而劲力精深的武功,精要之处,全在运劲发力,全凭劲强力猛取胜,当真是无坚不摧、无固不破,虽招数有限,但每出一掌均有龍吟虎嘯之勢、每出一招均具绝大的威力。

降龍十八掌原為廿八掌,從創幫之主傳承自汪劍通再傳到蕭峰時,因為後十招過於繁瑣,且威力卻遠不如前十八掌,經虛竹和蕭峰刪除重複後,威力更勝一籌,在蕭峰死後由虛竹把降龍十八掌代傳下任幫主,又輾轉授至洪七公之手。

由於降龍十八掌並非只傳幫主繼承人,所以洪七公也教給了郭靖和立有大功的黎生一招「神龍擺尾」,甚至郭靖也曾傳授給武敦儒、武修文兄弟,在第一版的倚天屠龍記中武家後人武烈也有降龍十八掌的部分秘笈,謝遜也曾教給張無忌些許降龍十八掌的招式,但是在後來的改版中這些內容都刪去。當郭靖女婿耶律齊接任丐幫幫主後,郭靖亦授他降龍十八掌,但因為襄陽被攻陷時,郭家眾人多殉國身亡,耶律齊後任的幫主並沒學全,只練成其中十四掌,後來傳到史火龍時只剩十二掌,在他被成昆殺害後,這門掌法似乎便告失傳,笑傲江湖時代的解風明顯不會這門掌法。

降龍十八掌的大部分招式的名字由易經而來,包括:
\begin{enumerate}
\item 亢龍有悔(乾卦上九): 其招式为左腿微屈,右臂内弯,右掌划一圆圈,向外推去。
\item 飛龍在天(乾卦九五): 这一招必先跃起半空,居高下击,才能显见奇大的威力,它一定要配合轻功跳跃之技,由上而下给予敌人痛击,算起应该是一种技巧性较高的武技。
\item 龍戰於野(坤卦上六): 左臂右掌,均是可虚可实,非拘一格。用虚实相生,阴阳相参的手法扰乱对方,自己则可以趁虚而入,是一式诱敌策。
\item 潛龍勿用(乾卦初九): 则是右手屈起食中二指,半拳半掌,向敌人胸口打去,左手同时向里钩拿,右推左钩,让敌人难以闪避。这是一种左右夹击的攻势,让人无处可避,尽在自己的掌握之中。
\item 利涉大川(大畜、同人、未济等卦多次出现): 逼退敌人之招式,用意在于使敌人勿近其身,以为自保。
\item 鴻漸於陸(漸卦九三): 此两招是一种技术性的逼退敌人之招式,用意在于使敌人勿近其身,以为自保。
\item 突如其來(離卦九四)
\item 震驚百里(震卦彖辭): 双掌向前平推,这是降龙十八掌中威力极大的一招。
\item 或躍在淵(乾卦九四): 先提一口气,然后以气化掌,左掌前探,右掌嗖的从左掌下穿了出去,直击对手小腹。是属于一种至刚至阳的正面攻势。
\item 神龍擺尾(原名履虎尾,履卦九四): 此乃降龙十八掌救命绝招,任何有生命的东西在他有生存受到威胁时所生成而出的反应力道,自是非同小可。
\item 見龍在田: 这一招是用于狭小空间的防身之术,它或为缓冲高手绵密不绝的攻势之用。
\item 雙龍取水(乾卦彖辭): 这一招转守为攻之策,当手腕被别人擒拿,此招可以顺腕翻过,以又重又快的掌法拍击敌人肩头,是一招转守为攻的法门。
\item 魚躍於淵(小畜卦辞): 此招由下而上的攻敌之术,与飞龙在天相为反生,是一种败中求胜之道。
\item 時乘六龍(乾卦九二)
\item 密雲不雨(損卦彖辭)
\item 損則有孚(坤卦初六)
\item 履霜冰至(大壯上六): 肘往上微抬,右拳左掌,直击横推,一快一慢的打出去。掌法之中刚柔并济,正反相成,实是妙用无穷,为“降龙十八掌”中较为阴柔的一技。
\item 羝羊觸藩(震卦初九): 意欲以掌力内功和着全身的体重,以快速的步伐,让敌人避无可避射无可射,其姿态就如一只受到刺激的羊一样,不顾一切地想冲出栅栏,威力相当惊人。
\end{enumerate}

\section{出入}
該武功在金庸小說中有出入。在《射鵰英雄傳》初版中,洪七公曾說這些有一半是自創的,但是在《天龍八部》中,丐幫幫主喬峰學會了全部十八式,且至死都未收徒授藝,前後矛盾。後來在第三版時才修正成洪七公僅為傳承者,並無自創招式。\footnote{順帶一提的是,丐幫之寶打狗棒也在喬峰任上遺失,但後來洪七公卻手持打狗棒,甚至與「西毒」歐陽鋒對戰,最後還傳給黃蓉。此漏洞亦於第三版時修正。}
\section{改編自降龍十八掌的武功}
由於金庸所創造武功「降龍十八掌」之名氣甚大,因此在之後出現了許多參考或托名「降龍十八掌」的武功。
\subsection{武狀元蘇乞兒}
降龍十八掌亦在周星馳主演的電影《武狀元蘇乞兒》中出現,與金庸小說的版本相似,這門武功同屬於丐幫幫主相傳的。

\begin{multicols}{3}
  \begin{enumerate}
  \item 飛龍在天
  \item 神龍擺尾
  \item 黑龍偷心
  \item 雙龍出海
  \item 見龍在田
  \item 龍飛鳳舞
  \item 伏虎降龍
  \item 縮龍成寸
  \item 龍蛇混雜
  \item 龍的傳人\footnote{是周星馳其他電影的名稱}
  \item 龍鳳呈祥
  \item 龍馬精神
  \item 望夫成龍\footnote{是周星馳其他電影的名稱}
  \item 殺龍有悔
  \end{enumerate}
\end{multicols}
%%% Local Variables: 
%%% mode: latex
%%% TeX-master: "../sample"
%%% End: 
