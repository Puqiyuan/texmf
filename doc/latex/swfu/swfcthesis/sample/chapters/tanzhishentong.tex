\chapter{弹指神通}
彈指神通\footnote{参见维基百科 - \href{http://zh.wikipedia.org/wiki/\%E5\%BC\%B9\%E6\%8C\%87\%E7\%A5\%9E\%E9\%80\%9A}{彈指神通}}為是金庸武俠小說中黃藥師所創的武功。

右手中指曲起,扣在拇指之下彈出,手法精微奧妙,射程甚遠,速度勁急之極,力道強勁異常,破空之聲異常響亮,
可用於彈落敵兵器,及彈敵穴道,力道亦強勁之極。

\section{簡介}
此功精微奧妙,唯有创始人「東邪」黃藥師的劈空掌、「西毒」歐陽鋒的蛤蟆功、「南帝」段智興的一陽指、「北丐」洪七公、「北俠」郭靖的降龍十八掌、「西狂」楊過的黯然銷魂掌可以比擬,也只比「中神通」王重陽的先天功稍遜一籌。

《射鵰英雄傳》中,黃藥師與周伯通比玩石彈、在歸雲莊彈石指點梅超風,都是使的這門功夫。

《神鵰俠侶》中,黃蓉亦以此功彈酒杯。黃藥師更把此功及玉簫劍法傳與楊過以剋制李莫愁的五毒神掌及拂塵

此外,《倚天屠龍記》的明教「光明左使」楊逍和《俠客行》的「摩天居士」謝煙客也會此功。

其實原著中並未提及此技為黃藥師所創,但由於黃藥師聰明絕頂,故許多讀者皆認為是其所創。
\section{電視劇版本}
\begin{itemize}
\item 1982年的電視劇楚留香中主角楚留香會彈指神通。在古龍小說中楚留香的絕技是彈指神功?(小說中應該未曾提及楚留香會彈指神功)
\item 2007年的電視劇強劍中主角成風示範彈指神通,但被叫作彈指神功。
\end{itemize}
\section{槍神版本}
2006年的RPG槍神中主角白虎會彈指神通。

%%% Local Variables: 
%%% mode: latex
%%% TeX-master: "../sample"
%%% End: 
