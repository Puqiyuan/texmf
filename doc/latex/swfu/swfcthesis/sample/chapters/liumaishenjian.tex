\chapter{六脈神劍}
六脈神劍
\footnote{参见维基百科 -
  \href{http://zh.wikipedia.org/wiki/\%E5\%85\%AD\%E8\%84\%88\%E7\%A5\%9E\%E5\%8A\%8D}{六脈神劍}}是
金庸武俠小說中的一套武功,使用者為《天龙八部》主角之一—段譽,堪稱大理段氏最厲害的武功, 讓段譽面對敵
手時幾乎所向無敵,但因段譽對武功的瞭解不夠透徹,實戰經驗也不足,常面臨時靈時不靈的情況, 威力也有所折扣。

\section{簡介}
六脈神劍,並非真劍,乃以渾厚內力的指力發出六種內力,含於指尖的內力隔空激發出去,使其以極高速在空中運動的一門技術,做架簡單,功效卓著,感應強烈,均為首屈一指,久習可得奇效。達到指劍的境界,即指力所能及的地方,有如有一柄無形的劍,無論是橫掃或虛指,均可傷敵,劍氣有質無形,出劍時急如電閃,迅猛絕倫,交叉運用,以氣走劍殺人於無形,異常神奇,堪稱劍中無敵,可稱無形氣劍。

它為大理段氏天龍寺鎮寺之寶,不傳段氏俗家子弟,只有天龍寺僧人,方蒙傳授。六脈神劍源自於人體中十二經脈
中的六脈—手太陰肺經、手陽明大腸經、手少陰心經、手少陽三焦經、手厥陰心包經、手太陽小腸經。

\section{六脈運行}

\subsection{右手剑}
\begin{itemize}
\item 右手拇指─少商劍〈特點:劍路雄勁,頗有石破天驚,風雨大至之勢〉:
中府→天府→尺澤→孔最→列缺→經渠→太淵→魚際→少商,屬於「手太陰肺經」。
\item 右手中指─中衝劍〈特點:大開大闔,氣勢雄邁〉:
天池→天泉→曲澤→郄門→間使→大凌→勞宮→中衝,屬於「手厥陰心包經」。
\item 右手小指─少衝劍〈特點:輕靈迅速〉:
極泉→青靈→少海→靈道→通里→陰郩→神門→少府→少衝,屬於「手少陰心經」。
\end{itemize}
\subsection{左手剑}
\begin{itemize}
\item 左手食指─商陽劍〈特點:巧妙靈活,難以捉摸〉:
迎香→扶突→天鼎→肩與→曲池→手三里→陽溪→合谷→商陽,屬於「手陽明大腸經」。
\item 左手無名指─關衝劍〈特點:以拙滯古樸取勝〉:
絲竹空→耳門→翳風→肩髎→天井→支溝→外關→陽池→中渚→液門→關衝,屬於「手少陽三焦經」。
\item 左手小指─少澤劍〈特點:忽來忽去,變化精微〉:
聽宮→顴髎→天容→天窗→肩中俞→秉風→天宗→臑俞→小海→支正→養老→腕骨→后谿→少澤,屬於「手太陽小腸經」。
\end{itemize}

\section{六脈神劍經}
大理天龍寺的鎮寺之寶,大理國祖師爺段思平所創,大理段氏的至高無上武功。

故老相傳,其與易筋經為武林兩大不世瑰寶,經上記載六脈神劍的修鍊方法,總共有六張圖譜,每幅圖上的劍氣圖都是縱橫交叉的直線、圓圈和弧形。

由於六脈神劍不傳段氏俗家子弟,所以連段正淳等人也不知曉《六脈神劍經》藏於天龍寺。

在吐蕃國師鳩摩智強行取經的事件中,枯榮為了不讓經書落入外人手中,以「一陽指」內力將圖譜焚毀。

由於枯榮等六人各學得一路劍法,段譽也將六式劍招全學齊,讓六脈神劍不致失傳。

%%% Local Variables: 
%%% mode: latex
%%% TeX-master: "../sample"
%%% End: 
