\documentclass{swfcthesis}

\addbibresource{tutorial.bib}    % 参照教程自己去写一个.bib文件

\begin{document}

\Title{论文标题}	% 论文标题
\Author{作者姓名} 	% 作者姓名 
\Advisor{指导教师姓名}      %指导教师姓名
\AdvisorTitle{指导教师职称} %指导教师职称
\AdvisorInfo{指导教师简介(约百余字)}	%指导教师简介(约百余字)
\Month{六}		    %月份(比如 六)
\Year{二〇一二} 	    %年份(比如 二〇一二)
\Univ{西南林业大学}    		%校名
\School{计算机与信息科学学院}		%院系名称
\Subject{计算机科学与技术专业}	%专业名称(比如 电子信息工程专业)
\Docname{本科毕业(设计)论文}	%本科?研究生?
\Abstract{这里写论文摘要(约两百字)}	%论文摘要(约两百字)
\Keywords{这里写关键字,比如 电阻, 电容}  %关键字(比如 电阻, 电容)
\Acknowledgments{这里写鸣谢(约百余字)} %鸣谢 (感谢人民感谢党,约百字)
\enTitle{英文标题}			%英文标题
\enAuthor{英文姓名}		%作者英文姓名
\enUniv{Southwest Forestry University} %英文校名
\enSchool{School of Computer and Information Science} %英文院系名称
\enAbstract{英文摘要}    	 %论文英文摘要
\enKeywords{英文关键字}	 %英文关键字

%%% 下面六行不要动!
\makepreliminarypages % 排版封面
\frontmatter          
\tableofcontents      % 输出目录
\listoffigures        % 插图目录
\listoftables         % 表格目录
\mainmatter

\chapter{第一章标题}
\label{cha:one}


正文部分从此开始。可以参考模版目录中的各章节tex文件来写。

\section{第一节标题}
\label{sec:one}

Hello, world!

\subsection{第一小节}
\label{sec:whatever}

Hello again, world!

\subsubsection{第一小小节}

again?

\chapter{又一章标题}
\label{cha:two}

接着写吧接着写吧接着写吧接着写吧

%%% 正文部分到此结束。
%%% 下面是『参考文献』、『指导教师简介』、『鸣谢』、『附录』

%% 不要动下面四行!
\Appendix{}
\printbibliography[heading={bibintoc},title={参考文献}] % 输出参考文献
\advisorinfopage{}                 % 输出指导教师简介
\acknowledgmentspage{}             % 输出鸣谢

%%% 下面是附录部分,可以没有。

\chapter{我也不知道为什么要写附录} %附录一
\label{app:one}

可以参考模版目录中的 appendix.tex 文件来写。

\chapter{我居然编程了!} %附录二
\label{app:two}

% 插入程序代码
\inputminted[fontsize=\small]{c}{hello.c}

% 也可以这样
\begin{listing}[H]
  \inputminted{c}{hello.c}
  \caption{我居然编程了!}
  \label{lst:hello}
\end{listing}  

% 不要动下面几行!

\end{document} % 此行后面不要再写任何文字

%%% Local Variables:
%%% mode: latex
%%% TeX-master: t
%%% End:
